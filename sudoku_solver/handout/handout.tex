\documentclass{article}

% Fonts
\usepackage{fontspec}

% General formatting
\usepackage[margin=1.3in, headheight=3ex, headsep=3ex]{geometry}
\usepackage{fancyhdr}
\usepackage{parskip}
\usepackage{amsmath}

% Colour
\usepackage[usenames,dvipsnames]{xcolor}

% Hyperlinks
\usepackage{url}
\usepackage{hyperref}

% Course details
\newcommand{\longcoursename}{
    Student Taught Courses (StuCo): Shilling the Rust Programming Language
}

\newcommand{\shortcoursename}{STUCO: RUSTLANG}
\newcommand{\courselocation}{PH A18B}
\newcommand{\meetingstarttime}{18:40}
\newcommand{\meetingendtime}{19:30}
\newcommand{\meetingdays}{Wednesday}
\newcommand{\longsemester}{Fall 2022}
\newcommand{\shortsemester}{F22}
\newcommand{\academicyear}{2021-22}
\newcommand{\deptcode}{98}
\newcommand{\coursecode}{008}
\newcommand{\fullcoursecode}{\deptcode-\coursecode}

% Headers and Footers
\pagestyle{fancy}
\lhead{Pierce; Duvall}
\rhead{\fullcoursecode\ \shortsemester}

\begin{document}
\thispagestyle{empty}
\begin{center}
    \begin{minipage}{.85\textwidth}
        \centering
        {\huge {\fullcoursecode} Homework 2: Sudoku Solver}

        \vspace{1em}

        \begin{tabular}{@{}rl@{}}
            Cooper Pierce & \href{mailto:cppierce@andrew.cmu.edu}{\texttt{cppierce@andrew.cmu.edu}} \\
            Jack Duvall & \href{mailto:jrduvall@andrew.cmu.edu}{\texttt{jrduvall@andrew.cmu.edu}} \\
        \end{tabular}

        \vspace{1em}

        \longsemester
    \end{minipage}
\end{center}

\section*{Overview}

The goal of this assignment is to give you a light introduction to how closures work in Rust, by having you write a simple continuation-based Sudoku solver. 

TODO write the rest of this

\section*{What You Will Write}

You will be writing code that runs the main solver recursion, building on existing Sudoku datastructures given to you.

This comprises editing the function \texttt{solve}, defined in \texttt{src/sudoku.rs}. We've set you up with a couple structs and enums in \texttt{src/types.rs} that you will be using extensively; reading that file or the output of \texttt{cargo doc} is recommended.

To test your code you can use the provided sudoku board files, as well as writing your own. We've provided a \texttt{main} function which wraps your code, reading from a board file passed on the command line, pretty-printing the board outputs. Like before, we've also incorporated some of the testing features in rustc/Cargo so you can automatically test your code on the provided files. Running \texttt{cargo test} will run these tests, which you can see at the bottom of \texttt{src/main.rs}.

\end{document}
