\documentclass{article}

% Fonts
\usepackage{fontspec}

% General formatting
\usepackage[margin=1.3in, headheight=3ex, headsep=3ex]{geometry}
\usepackage{fancyhdr}
\usepackage{parskip}
\usepackage{amsmath}

% Colour
\usepackage[usenames,dvipsnames]{xcolor}

% Hyperlinks
\usepackage{url}
\usepackage{hyperref}

% Course details
\newcommand{\longcoursename}{
    Student Taught Courses (StuCo): Shilling the Rust Programming Language
}

\newcommand{\shortcoursename}{STUCO: RUSTLANG}
\newcommand{\courselocation}{PH A18B}
\newcommand{\meetingstarttime}{18:40}
\newcommand{\meetingendtime}{19:30}
\newcommand{\meetingdays}{Wednesday}
\newcommand{\longsemester}{Fall 2022}
\newcommand{\shortsemester}{F22}
\newcommand{\academicyear}{2021-22}
\newcommand{\deptcode}{98}
\newcommand{\coursecode}{008}
\newcommand{\fullcoursecode}{\deptcode-\coursecode}

% Headers and Footers
\pagestyle{fancy}
\lhead{Pierce; Duvall}
\rhead{\fullcoursecode\ \shortsemester}

\begin{document}
\thispagestyle{empty}
\begin{center}
    \begin{minipage}{.85\textwidth}
        \centering
        {\huge {\fullcoursecode} Homework 1: PPM Parsing}

        \vspace{1em}

        \begin{tabular}{@{}rl@{}}
            Cooper Pierce & \href{mailto:cppierce@andrew.cmu.edu}{\texttt{cppierce@andrew.cmu.edu}} \\
            Jack Duvall & \href{mailto:jrduvall@andrew.cmu.edu}{\texttt{jrduvall@andrew.cmu.edu}} \\
        \end{tabular}

        \vspace{1em}

        \longsemester
    \end{minipage}
\end{center}

\section*{Overview}
The goal of assignment is to get you used to basic Rust constructs and syntax, filling out the core of a program that actually does something useful! We hope this will give you familiarity with Rust's development workflow and make you appreciate the language.

This assignment will have you write a basic parser for a simple image format known as PPM.

\section*{PPM Images}
PPM is a very simple image format, consisting of just:
\begin{itemize}
    \item A ``magic number" to distinguish it from other files
    \item Width and height of the image
    \item The maximum intensity value of
    \item Pixels as packed RGB pixel values, read sequentially row-by-row from the top left of the image.
\end{itemize}

\url{http://ailab.eecs.wsu.edu/wise/P1/PPM.html} outlines the format; for simplicity, we'll summarize it again here. Note that we'll be using the binary format, where each pixel channel takes up exactly one byte.

\subsection*{PPM Format Specification}

\begin{align*}
    \texttt{PPM} ::=&\ \texttt{MagicHeader}\\
                    &\ \texttt{CommentsAndWhitespace}\ \texttt{Width}\\
                    &\ \texttt{CommentsAndWhitespace}\ \texttt{Height}\\
                    &\ \texttt{CommentsAndWhitespace}\ \texttt{Maxval}\\
                    &\ ``{\textbackslash}n"\ \texttt{Pixels}\\
    \texttt{MagicHeader} ::=&\ ``P6{\textbackslash}n"\\
    \texttt{CommentsAndWhitespace} ::=&\ (``{\textbackslash}t"\ |\ ``{\textbackslash}n"\ |\ ``{\textbackslash}x0C"\ |\ ``{\textbackslash}r"\ |\ ``\ "\ |\ (``\#"\ \texttt{Comment}\ ``{\textbackslash}n"))\\
    &\ \texttt{CommentsAndWhitespace}\\
    \texttt{Comment} ::=&\ ((\text{Any character except for } ``{\textbackslash}n")\ \texttt{Comment})\ |\ \epsilon\\
    \texttt{Width} ::=&\ ((``0"\ |\ ``1"\ |\ ``2"\ |\ ``3"\ |\ ``4"\ |\ ``5"\ |\ ``6"\ |\ ``7"\ |\ ``8"\ |\ ``9")\ \texttt{Width})\ |\ \epsilon\\
    \texttt{Height} ::=&\ \texttt{Width}\\
    \texttt{Maxval} ::=&\ \texttt{Maxval}\\
    \texttt{Pixels} ::=&\ \text{Exactly $w*h*3$ bytes, where $w$ and $h$ are the parsed width and height}
\end{align*}
How to read this table: groups being right next to each other means the text they match should be concatenated. The $|$ operator means that either the left or right character can be accepted. Characters in $``"$ means only that exact string is accepted. $\epsilon$ matches the empty string.

\subsection*{PPM Image Example}
An example would probably help understand the above specification a bit better:
\begin{verbatim}
P6
# I'm a PPM file! You can tell by the magic header
3 3 # This first number is the width, and the second is the height
# This next number is the maximum intensity of each pixel
# Exporting as "raw" in GIMP always gives 255, which makes sense; this is the
# maximum value of a u8. After this maxval, we are only allowed a single newline
# before the pixels start. These pixels can be arbitrary bytes!
255
<<<<<<<<<aaaaaaaaa~~~~~~~~~
\end{verbatim}
Copy-pasting this text into a file ending with \texttt{.ppm} and opening it in a compatible image-viewing program should give a 3x3 image with 3 horizontal gray stripes.

\section*{What You Will Write}
You will be writing code that parses the metadata in the PPM header, up until the pixels.

TODO rest of this

\end{document}